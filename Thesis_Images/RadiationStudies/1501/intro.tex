The need for extremely radiation tolerant electronics is one of the major issues confronting high energy physics in the era of High Luminosity running at the CERN \cite{CERN} Large Hadron Collider (HL-LHC).
Tests by Bonacini, $\textit{et al.}$ \cite{Bonacini} at CERN, published in 2012, established 65 nm CMOS as the leading candidate technology for HL-LHC electronics.  Using an X-ray beam, Bonacini, $\textit{et al.}$ exposed 65 nm transistors to a total dose of 200 Mrad.  Their results showed, with one exception, relatively small changes in transistor parameters for normal layout standard gate oxide thickness (core) transistors.  The exception was a dramatic loss of maximum drain-source current in the narrowest PMOS transistors.  The CERN group concluded that 65 nm CMOS technology could be used for HL-LHC applications with no special design considerations, except that all core devices should have width greater than 360 nm.

The RD53 collaboration was formed in 2014 to further explore the feasibility of using 65 nm CMOS technology to design a pixel readout chip for use at the HL-LHC \cite{RD53}.  
 The group established a total ionizing dose tolerance goal of 1 Grad.  The measurements reported in this paper were done in the context of RD53.  Discussions late in 2013 within RD53 centered on the fact that the data presented in reference \cite{Bonacini}, and also subsequent data collected by the CERN group and by a group from CPPM \cite{CPPM}, contain evidence of significant room temperature annealing during the time between X-ray exposures.  Both CMS and ATLAS currently plan to operate their HL-LHC pixel vertex detectors at approximately  $-20\,^{\circ}\mathrm{C}$.   
This choice is because the silicon strip trackers will operate at $-20\,^{\circ}\mathrm{C}$ in order to limit leakage current in the silicon sensors, which would otherwise require much more cooling and therefore more mass in the tracking volume.  Concern was expressed that because of reduced annealing, 65 nm circuits might experience greater radiation damage than had been observed in room temperature exposures if the circuits were maintained at  $-20\,^{\circ}\mathrm{C}$ during irradiation.

We report the results of an irradiation of 65 nm transistors performed using the Gamma Irradiation Facility\cite{GIF} at Sandia National Laboratories \cite{Sandia}.  The devices under test were maintained at a temperature $\lesssim -20\,^{\circ}\mathrm{C}$ during irradiation.