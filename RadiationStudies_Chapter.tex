%\section{\centering Experimental Setup}
\chapter{Radiation Tolerance of 65 nm CMOS Transistors}
%\chapter*{\centering Experimental Setup}
\label{ch:RadStudies}

The need for extremely radiation tolerant electronics, especially in the era of High Luminosity running at the LHC (HL-LHC), is a major issue confronting high energy physics. The HL-LHC will begin running in 2025 and the expected peak luminosity is $5\times10^{34}\; \mathrm{cm}^{-2}\mathrm{s}^{-1}$. At this luminosity, the particle flux near the collision vertex will be extremely high and the electronics in the pixel detector will need to be able to operate while accumulating a total ionizing dose of about 1 Grad. 

To lower the material density and power dissipation in the pixel detector, the plan for the HL-LHC readout chips in the pixel detector at CMS is to upgrade from the current 130 nm complementary metal-oxide-semiconductor (CMOS) technology to 65 nm CMOS technology. Previous studies~\cite{CMOSXrayRadiation} showed that the properties of 65 nm CMOS technology did not dramatically change after being exposed to a total dose of 200 Mrad. However, these studies were conducted at room temperature and the pixel detector will be operated at $-20^{\circ}\mathrm{C}$ to limit the leakage current in the silicon strip trackers. At a lower temperature, the CMOS devices will not anneal as much and the radiation damage might be greater than had been observed in room temperature exposures. Thus, it is important to characterize the response of 65 nm CMOS technology to large radiation doses while operating at $-20^{\circ}\mathrm{C}$.

This chapter summarizes an experiment~\cite{CMOSRadiation} that characterized the response of 65 nm CMOS transistors to a cumulative radiation dose of 1 Grad while being held below $-20^{\circ}\mathrm{C}$.

\section{Experimental Details}

\subsection{Transistor Test Setup}

A 65 nm CMOS Application Specific Integrated Circuit (ASIC) containing individual transistors connected to wire bond pads was built for radiation tolerance testing. Transistors within the ASIC were laid out in groups of similar transistors, for example all N-type metal-oxide-semiconductor (NMOS) transistors with the same channel length (L). Within a group, all transistors share a gate pad and a source/drain pad. The other drain/source of the transistor is connected to its own wire bonding pad. The devices tested were P-type metal-oxide-semiconductor (PMOS) and NMOS core transistors operated at 1.2 V and NMOS input/output (I/O) transistors with double thickness gate oxide operated at 2.5 V. Several transistor sizes were included for core PMOS and NMOS transistors: transistors with L $=60$ nm and a channel width (W) between 120 and 1000 nm, one transistor with W $=500$ nm and L $=500$ nm, and one with W $=5000$ nm and L $=5000$ nm. The I/O NMOS transistors sizes tested had W $=280$ nm and L between 400 and 1000 nm, a transistor with W $=500$ nm and L $=500$ nm, and one with W $=5000$ nm and L $=5000$ nm.

The test ASICs were wire bonded into pin grid array (PGA) chip carriers so that they could be irradiated on simple printed circuit boards (PCBs) containing only sockets for the ASICs and connectors for bias voltage. During irradiation PMOS transistors were biased in two different ways: 

\begin{itemize}
\item The drains, sources, and gates were held at 1.2 V and the substrate was grounded. 
\item The gates and substrate were grounded while the drains and sources were held at 1.2 V. 
\end{itemize}
The NMOS core (I/O) transistors were biased with the gates held at 1.2 V (2.5 V) and all other nodes grounded. These are the worst-case bias conditions.

Transistor characteristics were measured by mounting a single chip carrier at a time on a different PCB test board containing switches that allow individual transistors to be measured independently. The test board was connected to two source measurement units (SMUs), one to bias transistor gates and one to measure drain-source currents. Characteristics were made by holding the core (I/O) transistors drain-source voltage at 1.2 V (2.5 V) and the drain-source current was measured as the gate-source voltage was swept from 0 to 1.2 V (2.5 V). 

\subsection{Irradiation Setup}

The irradiation of the test devices was performed at the Gamma Irradiation Facility (GIF) at Sandia National Laboratories.  The GIF uses $\ce{^{60}Co}$ sources to provide controlled doses of ionizing radiation. $\ce{^{60}Co}$ decays by beta decay to an excited state of $\ce{^{60}Ni}$ which then relaxes to the ground state by emitting two gamma rays of energy 1.17 and 1.33 MeV. The $\ce{^{60}Co}$ is held in stainless steel “source pins” so that none of the beta electrons escape the steel source pins. 40 pins are mounted in a straight-line array that is held at the bottom of an 18 foot deep pool of deionized water to provide shielding when not in use and raised out of the water when an irradiation takes place.

The test ASICs were held inside stainless steel thermos bottles positioned approximately two inches from the face of the array of pins. Cooling was provided by vortex tube coolers mounted in holes drilled through the plastic thermos bottle lids. Figure~\ref{fig:Thermos} shows the thermos bottle assembly. To maintain the temperature of the thermos bottles, which were heated by the gamma rays interacting with the walls of the thermos bottles, the compressed air that was input to the vortex tubes was precooled and passed through insulated copper tubes. The temperature within the thermos bottles was measured and recorded by a K-type thermocouple in each thermos bottle. Figure~\ref{fig:Temperature} shows the temperature of two thermos bottles during the irradiations.

\begin{figure}[htbp]
\begin{center}
\includegraphics[width=4 in]{RadiationStudies/thermos.pdf}
\end{center}
\caption{Pictures of the thermos bottle, including an irradiation printed circuit board with four chip carriers, before insertion of the irradiation board into the thermos bottle. On the left, the red arrow points to the vortex tube on top of the thermos bottle lid. On the right, the red arrow points to an antistatic bag which wraps the irradiation board and low-voltage cable before irradiation. These bags keep the boards and voltage cables dry during the irradiation.}
\label{fig:Thermos}
\end{figure}

\begin{figure}[htb!]
\begin{center}
\includegraphics[width=5.5 in]{RadiationStudies/comp_temp.pdf}
\end{center}
\caption{The temperature measured inside the two thermos bottles during the long irradiations. No irradiation was performed on June 8 or 9. The two spikes where the temperature reached about $8^{\circ}\mathrm{C}$ in both thermos bottles for 30 minutes on June 12 occurred because the compressed air unexpectedly turned off.}
\label{fig:Temperature}
\end{figure}

The does rate the test ASICs received was 1425 rad/s and was measured by an ion chamber placed inside of a thermos bottle. The uniformity of the radiation field was checked by irradiating thermoluminescent dosimeters (TLDs) taped to each of the chip carriers on the irradiation PCB. The TLDs also provided a second measurement of the dose rate.

Twelve irradiations were performed over 15 days, as show in Table~\ref{tab:IrradiationSchedule}, and after each irradiation step a single characteristic curve was recorded for each transistor. All measurements were made at room temperature, but when a test ASIC wasn't being irradiated or measured they were stored at $-20^{\circ}\mathrm{C}$ in a freezer. After the full irradiation, devices were kept at room temperature for a week and multiple characteristics were taken to characterize the annealing effects. The transistors were then held in an oven at $100^{\circ}\mathrm{C}$ for one week to simulate an extended annealing period and a final set of measurements was made.

\begin{table}
\begin{center}
\caption{The irradiation schedule, showing the 2 weeks it took to accumulate 1 Grad.}
\begin{tabular}{| p{2cm} | p{2.5cm} | p{2cm} | p{4cm} |}
\hline
Date & Length & Dose(Mrad) & Cumulative Dose(Mrad)\\ \hline
June 2 & 1 hour & 5 & 5\\ \hline
June 3 & 1 hour & 5 & 10 \\ \hline
June 3 & 1 hour 45 mins & 9 & 19 \\ \hline
June 3 & 4 hour 15 mins & 22 & 41 \\ \hline
June 4-5 & 12 hours & 62 & 103 \\ \hline
June 5-6 & 22 hours & 113 & 215 \\ \hline
June 6-7 & 22 hours & 113 & 329 \\ \hline
June 9-10 & 22 hours & 113 & 441 \\ \hline
June 10-11 & 17 hours & 87 & 528 \\ \hline
June 11-12 & 22 hours & 113 & 641 \\ \hline
June 12-13 & 22 hours & 113 & 754 \\ \hline
June 13-16 & 66 hours & 339 & 1093 \\
\hline
\end{tabular}
\label{tab:IrradiationSchedule}
\end{center}
\end{table}


\section{Analysis}

Two quantities were extracted from each transistor characteristic: the maximum drain-source current and the threshold voltage, $V_{th}$. The quadratic extrapolation method was used to determine the threshold voltage~\cite{QuadraticMethod}. As shown in Figure~\ref{fig:QuadraticMethod}, $V_{th}$ is defined to be the voltage at which a line tangent to the curve $\sqrt{|I_{ds}|}$ vs $V_{gs}$ at the point of maximum $\frac{d \sqrt{|I_{ds}|}}{dV_{gs}}$ intercepts the $I_{ds}=0$ axis. The slope of the curve was determined by fitting it with a fifth order polynomial and differentiating the fit function. 

\begin{figure}[htb!]
\begin{center}
\includegraphics[width=5 in]{RadiationStudies/Quadratic_Method.pdf}
\end{center}
\caption{This figure illustrates the quadratic extrapolation method used to determine the threshold voltage ($V_{th}$) of an NMOS transistor. The blue data points are the transistor characteristic and the red ones are computed using finite differences $\frac{\sqrt{I_{ds}(N+1)}-\sqrt{I_{ds}(N)}}{V_{gs}(N+1)-V_{gs}(N)}$. The black curve is the result of differentiating the fifth order polynomial that was fit to the characteristic. $V_{th}$ is the point on the $I_{ds}=0$ axis where the tangent to the characteristic intersects. For PMOS transistors, $|I_{ds}|$ is used since $I_{ds}$ is negative.}
\label{fig:QuadraticMethod}
\end{figure}

\section{Results}

Figure~\ref{fig:SuperpositionPlots}  illustrates the radiation effects observed in the data.  The most prominent effect is a decrease of the maximum drain-source current of core PMOS transistors.  The fractional decrease is largest for the smallest PMOS transistors and they decreased by more than a factor of two.  No significant difference was observed between the radiation-induced changes of PMOS transistors held at different bias voltages. This is illustrated in Figure~\ref{fig:PMOSBiasConditions}. The maximum drain-source current of core NMOS transistors also decreased, but only by $\sim5-10\%$.  No significant threshold shift was observed for any of the core transistors, but the threshold voltage of NMOS I/O transistors increased by 100 - 200 mV.  

%No error bars are included in the figures because the uncertainty in the SMU measurements is smaller than the symbols used to plot the measurements.

\begin{figure}[htb!]
%	\begin{minipage}[b]{.5\linewidth}
	\begin{subfigure}{.5\linewidth}
	\includegraphics[width=\linewidth]{RadiationStudies/pd012_2_comparison_paper.pdf}
	\end{subfigure}
%	\end{minipage}
%	\begin{minipage}[b]{.5\linewidth}
	\begin{subfigure}{.5\linewidth}
	\includegraphics[width=\linewidth]{RadiationStudies/pd036_2_comparison_paper.pdf}
%	\end{minipage}
%	\begin{minipage}[b]{.5\linewidth}
	\end{subfigure}
	\begin{subfigure}{.5\linewidth}
	\includegraphics[width=1\linewidth]{RadiationStudies/d024_1_comparison_paper.pdf}
	\end{subfigure}
%	\end{minipage}
	\hfill
%	\begin{minipage}[b]{.5\linewidth}
	\begin{subfigure}{.5\linewidth}
	\includegraphics[width=\linewidth]{RadiationStudies/ddg1u_1_comparison_paper.pdf}
	\end{subfigure}
%	\end{minipage}
\caption{Transistor characteristic curves for total dose up to 1.1 Grad of (upper left) a 120/60 core PMOS, (upper right) a 360/60 core PMOS, and for total dose up to 878 Mrad of (lower left) a 240/60 core NMOS, and (lower right) a 1000/280 2.5 V NMOS.}
\label{fig:SuperpositionPlots}
\end{figure}

\begin{figure}[htb!]
	\begin{subfigure}{.5\linewidth}
%\begin{minipage}[b]{0.5\textwidth}
	\centering
	\includegraphics[width=\linewidth]{RadiationStudies/pd012_comparing_bias_conditions_paper.pdf}
	\end{subfigure}
%\end{minipage}
\hspace{0.5cm}
%\begin{minipage}[b]{0.5\textwidth}
	\begin{subfigure}{.5\linewidth}
	\centering
	\includegraphics[width=\linewidth]{RadiationStudies/pd036_comparing_bias_conditions_paper.pdf}
	\end{subfigure}
%\end{minipage}
\caption{The change in maximum drain-source current for similar PMOS core transistors irradiated with different gate bias voltages. The graph on the left is for 120/60 transistors and the graph on the right is for 360/60 transistors.  The lines connecting points do not represent a fit, and are included only to make the plots easier to read.  The transistor characteristics measured for transistors in one of the test ASIC packages after 754 Mrad was accumulated were all offset by current not likely to have passed through the transistors (this can be seen in Figure~\ref{fig:SuperpositionPlots}).  Lines are not drawn through these points.  The most likely source of these offsets is leakage current due to moisture caused by condensation on the cold ASIC package.}
\label{fig:PMOSBiasConditions}
\end{figure}

Figure~\ref{fig:AnnealSuperpositionPlots} demonstrates the annealing effects observed in our data. Both the PMOS core transistors and the NMOS I/O transistors recovered significantly during the annealing period. 

\begin{figure}[htb!]
\begin{minipage}[b]{0.5\textwidth}
	\centering
	\includegraphics[width=\linewidth]{RadiationStudies/pd012_2_Anneal_comparison_paper.pdf}
\end{minipage}
\hspace{0.5cm}
\begin{minipage}[b]{0.5\textwidth}
	\centering
	\includegraphics[width=\linewidth]{RadiationStudies/ddg1u_1_Anneal_comparison_paper.pdf}
\end{minipage}
\caption{Transistor chararcteristic curves during the annealing period for (left) a 120/60 core PMOS and (right) a 1000/280 2.5 V NMOS.}
\label{fig:AnnealSuperpositionPlots}
\end{figure}

Figures~\ref{fig:MaxCurDrive_PMOS} and~\ref{fig:MaxCurDrive_NMOS} show the evolution of the maximum drain-source current for a representative selection of PMOS and NMOS core transistors during irradiation and annealing. Figure ~\ref{fig:DGNMOS_Vth} shows the threshold shift of a representative selection of NMOS I/O transistors during irradiation and annealing.

\begin{figure}[htb!]
\begin{minipage}[b]{0.5\textwidth}
	\centering
	\includegraphics[width=\linewidth]{RadiationStudies/Comparing_MaxCurrentDrive_PMOS_paper.pdf}
\end{minipage}
\hspace{0.5cm}
\begin{minipage}[b]{0.5\textwidth}
	\centering
	\includegraphics[width=\linewidth]{RadiationStudies/Comparing_MaxCurDrive_Anneal_PMOS_paper.pdf}
\end{minipage}
\caption{The graph on the left shows the loss of maximum drain-source current during irradiation for 4 PMOS core transistors. The graph on the right shows the recovery of maximum drain-source current for the same 4 transistors during and after annealing.
As in Figure 6, lines are included to make the plots easier to read.
Once again, lines are not drawn through the points corresponding to measurements made after 754 Mrad of transistors in one of the ASIC packages.}
\label{fig:MaxCurDrive_PMOS}
\end{figure}

\begin{figure}[htb!]
\begin{minipage}[b]{0.5\textwidth}
	\centering
	\includegraphics[width=\linewidth]{RadiationStudies/Comparing_MaxCurrentDrive_NMOS_paper.pdf}
\end{minipage}
\hspace{0.5cm}
\begin{minipage}[b]{0.5\textwidth}
	\centering
	\includegraphics[width=\linewidth]{RadiationStudies/Comparing_MaxCurDrive_Anneal_NMOS_paper.pdf}
\end{minipage}
\caption{The graph on the left shows the loss in maximum drain-source current after each irradiation step for 9 NMOS core transistors. The graph on the right shows the change in maximum drain-source current for the same 9 transistors during and after annealing.}
\label{fig:MaxCurDrive_NMOS}
\end{figure}

\begin{figure}[htb!]
\begin{minipage}[b]{0.5\textwidth}
	\centering
	\includegraphics[width=\linewidth]{RadiationStudies/Comparing_ThreshVolt_DGNMOS_paper.pdf}
\end{minipage}
\hspace{0.5cm}
\begin{minipage}[b]{0.5\textwidth}
	\centering
	\includegraphics[width=\linewidth]{RadiationStudies/Comparing_ThreshVolt_Anneal_DGNMOS_paper.pdf}
\end{minipage}
\caption{The shift in threshold voltage for 8 NMOS I/O transistors irradiated to 878 MRad is shown in the graph on the left, while the graph on the right shows $V_{th}$ for the same 8 transistors during and after annealing.  No significant annealing was observed for the two zero $V_{th}$ I/O transistors.}
\label{fig:DGNMOS_Vth}
\end{figure}

\section{Summary}

The irradiation of 65 nm CMOS transistors held at $-20^{\circ}\mathrm{C}$ was motivated by the need to simulate the actual operating conditions of the HL-LHC CMS pixel detector. The results show the same pattern of effects that had been observed at room temperature irradiations except the damage observed was less severe, rather than more severe. This could attributed to less atomic movement within the devices held at cold temperatures allowing the damage to not be as bad.

















