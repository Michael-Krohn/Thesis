%\section{\centering Experimental Setup}
\chapter{Introduction}
%\chapter*{\centering Experimental Setup}
\label{ch:Intro}

For millennia, humans have attempted to better understand the universe and explain phenomena that they observe within it. During this exploration, one of the questions that has always been asked is what are the fundamental constituents in nature? For most of our history, this question was answered through theoretical arguments. For example, in Ancient Greece, Lucretius and Democritus hypothesized that everything is composed of indivisible objects called “atoms.” This hypothesis was based on the empirical argument that matter is subject to irreversible decay and yet, new objects are created in nature, like a sapling sprouting from the ground, with identical properties to objects that have already decayed, like a fallen tree. Therefore, matter is made up of elements that are not visible to human senses and contain a substance’s properties. 

However, as a more rigid scientific method developed, experimental evidence was needed to validate a hypothesis. This eventually led to John Dalton providing the first experimental evidence for the existence of atoms in the early 19th century. He found that elements always react in ratios of small whole numbers and therefore, elements must be reacting in whole number multiples of discrete units of atoms. Then in the late 19th century and early 20th century, Thomson and Rutherford experimentally showed that an atom contains negatively charged particles and a positively charged nucleus. These revelations were needed for the field of modern particle physics, the study of the fundamental building blocks of matter and how they interact, to begin. 

As the 20th century continued, theoretical and experimental physics worked in unison to construct the Standard Model (SM) of particle physics. This theory put order to the numerous particles that had been discovered while also predicting the existence of new ones: the top quark, the tau neutrino, and the Higgs boson. Experimental physicists then set out to find these particles and this culminated with the 2012 discovery of the Higgs boson, a discovery that took place over 40 years after the original prediction. 

However, despite the great predictive success of the SM, the theory has its shortcomings: gravity is not incorporated in it, it does not predict neutrino oscillations, and it does not contain a viable dark matter candidate. To address these issues, many new theories have been proposed that extend or build upon the SM, called beyond the SM (BSM) theories. There are numerous experiments that search for hints of these theories and particle accelerators have taken a prominent role in these searches. The current leading particle accelerator complex in the world is the Large Hadron Collider (LHC), which is where the Higgs boson was discovered. Unfortunately, no evidence for new theories has been detected yet. Nevertheless, the LHC has only collected a small fraction of the data it is expected to throughout it’s lifetime. Therefore, as the LHC continues to run, physicists must innovatively examine the data for BSM physics.

This thesis outlines the search for a new, heavy resonance decaying into pairs of standard model Higgs bosons using the Compact Muon Solenoid (CMS) detector at the LHC. Chapter~\ref{ch:Theory} will layout the theoretical foundation for this search by defining the SM and the theories that predict heavy resonances. The LHC and the CMS detector will then be described in Chapter~\ref{ch:ExpSetup} followed by a summary of how events are reconstructed at CMS in Chapter~\ref{ch:ObjReco}. A brief aside will then be taken in Chapter~\ref{ch:RadStudies} to outline radiation studies done on complimentary metal-oxide-semiconductors that will be installed in an upgrade of the CMS detector. Chapter~\ref{ch:Analysis} will completely describe the heavy resonance search and finally, a summary and outlook will be given in Chapter~\ref{ch:Summary}.
